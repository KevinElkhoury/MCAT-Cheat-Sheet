\documentclass[10pt,landscape]{article}
\usepackage{multicol}
\usepackage{calc}
\usepackage{ifthen}
\usepackage[landscape]{geometry}
\usepackage{amsmath,amsthm,amsfonts,amssymb}
\usepackage{color,graphicx,overpic}
\usepackage{multirow}
\usepackage{array}
\usepackage{svg}
\usepackage{enumitem}
\usepackage{tikz}
\usepackage{tikz-3dplot}
\usepackage{pgfplots}
\usetikzlibrary{shapes.geometric}
\usetikzlibrary{decorations.markings}


\pdfinfo{
  /Title (PsychSoc.pdf)
  /Creator (TeX)
  /Producer (pdfTeX 1.40.0)
  /Author (Kevin)
  /Subject (PsychSoc Cheat Sheet)
  /Keywords (psychsoc, tex, latex)}

\geometry{top=.5in,left=.5in,right=.5in,bottom=.5in}

% Turn off header and footer
\pagestyle{empty}

% Redefine section commands to use less space
\makeatletter
\renewcommand{\section}{\@startsection{section}{1}{0mm}%
                                {-1ex plus -.5ex minus -.2ex}%
                                {0.5ex plus .2ex}%
                                {\normalfont\large\bfseries}}
\renewcommand{\subsection}{\@startsection{subsection}{2}{0mm}%
                                {-1explus -.5ex minus -.2ex}%
                                {0.5ex plus .2ex}%
                                {\normalfont\normalsize\textit}}
\renewcommand{\subsubsection}{\@startsection{subsubsection}{3}{0mm}%
                                {-1ex plus -.5ex minus -.2ex}%
                                {1ex plus .2ex}%
                                {\normalfont\small\underline}}
\makeatother

\renewcommand{\baselinestretch}{1.5}

% Don't print section numbers
\setcounter{secnumdepth}{0}


\setlength{\parindent}{0pt}
\setlength{\parskip}{0pt plus 0.5ex}

%My Environments
\newtheorem{example}[section]{Example}


\pgfplotsset{width=6cm,compat=1.9}

% -----------------------------------------------------------------------

\begin{document}
\raggedright
\footnotesize
\begin{multicols}{3}


% multicol parameters
% These lengths are set only within the two main columns
%\setlength{\columnseprule}{0.25pt}
\setlength{\premulticols}{1pt}
\setlength{\postmulticols}{1pt}
\setlength{\multicolsep}{1pt}
\setlength{\columnsep}{2pt}

\begin{center}
     \Large{\textbf{PsychSoc}} \\
\end{center}

\section{Culture and Society}
\subsection{Terminology}
\textbf{Culture:} Rules, norms, and traditions passed on from one generation to the next\newline
\textbf{Society:} Structures in a population including groups and institutions \newline
\textbf{Culture Shock:} Feelings of uncertainty, fear, and disorientation when encountering a new or different culture \newline
\textbf{Ethnocentrism:} Tendency to view your own culture as the best and judge other cultures accordingly \newline
\textbf{Cultural Relativism:} Understanding there are multiple cultures and to not judge another cultures based off of your own

\subsection{Types of Culture}
\textbf{Subculture:} A culture of a meso-level (medium size) community within a larger community sharing some aspects of their culture \newline
\textbf{Micoculture:} Small cultures that affect just a small part of someones life (ex: girl scouts, boarding schools...) \newline
\textbf{Counterculture:} A conflict between a subculture and a dominant culture

\subsection{Rules of Culture}
\begin{enumerate}[nolistsep]
\item People share a culture in society
\item Culture is adaptive
\item Culture builds on itself
\item Culture in transmitted from one generation to the next
\end{enumerate}

\subsection{Culture Lag}
Often times culture is slow to catch up to technology:
\textbf{Material Culture:} Physical and technological aspects of our lives \newline
\textbf{Non-Material Culture:} Ideas, beliefs, and values which progress slower than material culture

\subsection{Evolution and Culture}
Just like physical traits undergo natural selection behaviors can as well:
\begin{itemize}[nolistsep]
\item Medicine
\item Marriage
\item Death rituals/Funerals
\item Language
\end{itemize}
All groups share these behaviors so they were likely favored through natural selection


\section{Demographics}
\subsection{Stucture of Society}
\subsubsection{Age}
\textbf{Dependency Ratio:} People who are economically dependent to people to who economically independent \newline
\textbf{Life Course Theory:} Aging is a biological, psychological, and social process \newline
\textbf{Age Stratification Theory:} Behvior is based off of our age group \newline
\textbf{Age Activity Theory:} As people age they lose social interactions which need to be replaced to maintain morale \newline
\textbf{Disengagement Theory:} As you age you seperate from society \newline
\textbf{Continuity Theory:} As you age you try to make changes to keep your life constant 

\subsubsection{Race and Ethnicity}
\textbf{Race:} Socially defined category based off of physical differences \newline
\textbf{Ethnicity:} Socially defined category based off of cultural factors

\subsubsection{Sex, Gender, and Orientation}
\textbf{Sex:} Biological characteristics \newline
\textbf{Gender:} Social construction with 2 parts what someone identifies as and the gender one expresses outwardly \newline
\textbf{Orientation:} Who we are attracted to both in terms of pure attraction and sexual attraction

\subsection{Urbanization}
\subsubsection{Categorization}
\begin{enumerate}[nolistsep]
\item Rural
\item Exurbs
\item Suburbs
\item Urban
\item Metropolis
\item Megalopolis
\end{enumerate}

\subsubsection{Effects}
\textbf{Positives:} Wide variety of culture, and anonymity \newline
\textbf{Negatives:} Crowding

\subsubsection{Categories of People}
\textbf{Cosmopolites:} People looking for culture and utilities of the city \newline
\textbf{Single:} People looking for jobs, partners, and entertainment \newline
\textbf{Deprived and Trapped:} People who have no choice and cannot afford to get out

\subsection{Theories}
\textbf{Urban Decline:} As people move out of city centers buildings can start to be abandoned \newline
\textbf{Urban Renewal:} Revamping older and abandoned parts of a city \newline
\textbf{Dentrification:} When urban renewal caused an increase in property value pushing out the prior poorer population

\subsection{Population Dynamics}
Factors that determines growth rate and population size:
\begin{enumerate}[nolistsep]
\item Fertility
\item Mortality
\item Migration
\end{enumerate}

\subsection{Demographic Transition Model}
Stages of demographics:
\begin{enumerate}[nolistsep]
\item High birth rate due to economic benefits and high death rate
\item High birth rate but declining death rate due to increasing health care
\item Birth rate slowly increasining due to access to contraception with still declining death rates
\item Birth and death rate are low balancing each other out
\item Speculation: there may be a resource shortage forcing stabilizing population where growth rate will stop or reverse
\end{enumerate}

\subsection{Globilization}
\subsection{Theories}
\textbf{World System Theory:} World is divided into 3 types of countries
\begin{enumerate}[nolistsep]
\item Core countries: Wealthy stable countries with diverse economies and strong governments
\item Periphery countries: Poorer countries with weak governments typically reliant on exports of raw materials and easily influences by others
\item Semi-Peripherary Countries: Countries between core and periphery countries
\end{enumerate}
\textbf{Modernization Theory:} All countries follow  a simmilar path of development and less developed countries can follow the same route developed countries had taken \newline
\textbf{Dependency Theory:} 3rd world countries are integrated into the world system reliant on 1st world countries and are trapped and can't further develop \newline
\textbf{Hyperglobalist Perspective:} Globalization is a legitamate process where countries grow to depend on each other (not decided if good or bad) \newline
\textbf{Skeptical Perspective:} Globalization is not really happening and 3rd world countries are not actually being incorportated as 1st world countries are \newline
\textbf{Tranformational Perspective:} National governments are changing (very vague) \newline

\subsection{Trade and Transnational Corportations}
\textbf{International Trade:} Has been supported by regulatory groups and agreements often times benefitting private industries the most \newline
\textbf{Transnational Corportations:} Companies that spread throughout multiple countries for access to different markets and resources \newline
\textbf{Cheap Labor:} Developing nations may provide cheap labor and tax exemptions to promote companies settling in their nation to promote enconomic development \newline
\textbf{Diffusion:} International trade causes culture to be spread as well

\subsection{Social Movements}
\textbf{Activist vs Regressive Movement:} Movement trying to change society vs resist the change in society \newline
\textbf{Mass Society Theory:} People join social movements for a sense of community \newline
\textbf{Relative Depravation Theory:} Response to perceived inequality in rights or oppertunities and the beleif conventional methods will not help \newline
\textbf{Resource Mobilization Theory:} Social movement succeed and happen based off of availability and quality of resources and leadership. \newline
\textbf{Rational Choice Theory:} People make rational choices weighing all available options and choosing the best

\section{Social Strucutre Theories}
\textbf{Macro vs Micro Sociology:} Looking at how the big picture affects the individual and vice versa \newline
\textbf{Functionalism:} Society is heading towards an equilibrium where all parts play a role: social facts (norms, laws, etc...) and institutions  balance each other out as population grows and society changes\newline
\textbf{Manifest vs Latent Functions:} Manifest functions are the intended concequences of an institution and latent are the unintended concequences \newline
\textbf{Conflict Theory:} Associated with theories of Karl Marx where society is fighting over limited resources and that naturally society would change from feudelism to capitalism to socialism. Every society will have a norm of power (the thesis) and another side looking to change it (the anti-thesis) constantly in conflict \newline
\textbf{Feminist Theories:} Focus on gender inequalities against woman due to the patriarchy.

\subsubsection{Social Constructionism}
\textbf{Weak:} Most things have meanings because we gave it to them and are based off of our interactions with them ex: money, countries, borders, cultures however brute facts (laws of physics, biology...) still exist \newline
\textbf{Strong:} Everything is based off of social interactions there are no brute facts we created the idea of every fact \newline

\subsubsection{Symbolic Interactionism}
Looks at the a small scale view of society looking at individual interactions. The theory states that people assign meaning to things based off of interactions through 3 main tenants:
\begin{enumerate}[nolistsep]
\item We act based off the meaning we have given something
\item We give meanings to things based off out social interactions
\item The meaning we give something is not permanent but can change 
\end{enumerate}

\subsubsection{Rational Choice-Exchange Theory}
\textbf{Rational Choice Theory:}
Every action is based off of rational choices and can be used to explain society and its changes. We rank every choice and pick the best one for us based off 3 assumptions:
\begin{enumerate}[nolistsep]
\item Completeness: Every action can be ranked and one is always better
\item Transitivity: If choice A $>$ B and B $>$ C then A $>$ C 
\item Independence of relevant alternatives: If A$>$B$>$C introducing X will not change the order of the other variables so A$>$B$>$X$>$C
\end{enumerate}
\textbf{Exchange Theory:} Applying rational choice theory to interactions where we make interactions by weighing the pros and cons and choosing the best choice.

\section{Social Inequality}

\subsection{Social Mobility}

\subsubsection{Systems}
\textbf{Cast System:} System where you can only experience horizontal social movement \newline
\textbf{Class System:} A system where people can experience vertical social movement but starting in a certain class \newline
\textbf{Meritocracy:} A system where everyone stars as equals and experiences social movement purely based off of merit

\subsubsection{Generational Mobility}
\textbf{Intragenerational:} Mobility one experiences in their own lifetime \newline
\textbf{Intergenerational:} Social mobility changes accross generation for a group or family \newline
\textbf{Social Reproduction:} Social Inequalities tends to reproduce accross generations

\subsection{Poverty}
\textbf{Absolute Poverty Line:} Income required in order to survive \newline
\textbf{Relative Poverty:} Varying lined that is based as some value below the median income of a country where they can survive but not actively partake in society\newline

\subsection{Segregation}
\textbf{Concentration:} When a group is clustered to one area \newline
\textbf{Centralization:} A group clustered in the middle \newline
\textbf{Politics:} Segregated communities are politically weak posessing less voting power \newline

\subsection{Consciousness}
Concerning Marxist theories \newline
\textbf{Class Consciousness:} Working class realizing they have solidarity with one another and they can seize the means of production. \newline
\textbf{False Consciousness:} Working class is unable to see that they are being exploited


\section{Self and Society}

\subsection{Kohlberg Stages of Moral Development}
3 Stages in Morality:
\begin{enumerate}[nolistsep]
\item Pre-Moral (Focus on Self)
\begin{enumerate}[nolistsep]
\item Obedience vs Punishment
\item Individualism and Exchange
\end{enumerate}
\item Conventional (Focus on Society)
\begin{enumerate}[nolistsep]
\item Good boy and Good Girl
\item Law and Order
\end{enumerate}
\item Post-Conventional (Focus on the Individual)
\begin{enumerate}[nolistsep]
\item Social Contract
\item Universal Ethical Principle
\end{enumerate}
\end{enumerate}

\subsection{Social Influences}
\textbf{Imitation:} The act of copying another individual \newline
\textbf{Roles:} We act to fulfill a role that follows social norms \newline
\textbf{Reference Groups:} Groups that individuals refer when evaluating oneself \newline
\textbf{Culture and Socialization:} Contributions of our society, culture, and enviroment

\subsection{Theories}
\subsubsection{I and Me} 
\textbf{I:} Our views on society \newline
\textbf{Me:} Societies view \newline
We exist as a pairing of the I and Me. We go through stages before we get their however:
\begin{enumerate}[nolistsep]
\item Prepatory Stage: Imitate other
\item Play Stage: Takes on roles by pretending and playing games understanding their views in that role
\item Game stage: Final stage, understands that people have multiple roles and opinions 
\end{enumerate}

\subsubsection{Looking Glass Self}
How we view ourselves is determined through 3 steps:
\begin{enumerate}[nolistsep]
\item Imagine how we appear to others
\item Thing what other must thing of our appearence
\item Revise our opinions of ourselves
\end{enumerate}

\subsubsection{Dramaturgical Approach}
\textbf{Fronst Stage:} The impression we give off to people and how we act in public \newline
\textbf{Back Stage:} Our private lives, how we act in close communities and in private

\section{Perception}
\subsection{Social Perception}
\textbf{Just World Phenomenon:} People think universe is fair so people get what they deserve. People use this to justify their actions. \newline
\textbf{Self-serving bias:} Our success is due to internal factors while our failures
are due to external factors. \newline
\textbf{Fundamental Attribution Error:} Failure of others are due to internal factors while our failures are due to our situation. \newline
\textbf{Basic Covariation Model:} If someone does something consistently we think it to be an internal reason while if something is done rarely or by a large amount of people it is due to a situational reason. \newline
\textbf{Primacy Bias:} First impression weighed more than other impressions \newline
\textbf{Recency Bias:} More recent impressions are weighed more than other ones \newline
\textbf{Halo Effect:} Tendency for positive impressions to cause us to perceive their other traits to be better than they actually are. The opposite can happen with negative impressions (Devil Effect).


\subsection{Stereotype}
\textbf{Stereotyping:} Attributing a certain trait to a group of individuals  (Cognition) \newline
\textbf{Prejudice:} An opinion about a group formed to due to a sterotype (affective) \newline
\textbf{Discriminion:} Acting in a certain way due to a prejudice (behavior) \newline
\textbf{Self-Fulfilling Prophecy:} Something becoming true due to a belief \newline

\subsection{What causes Prejudice?}
\textbf{Cognition:} Some personality traits can be more vulnerable for example authoritarian personality type \newline
\textbf{Emotion:}
\begin{enumerate}[nolistsep]
\item Frustration-Aggresion Hypothesis: People misplacing frustration towards minority group to avoid targeting it towards people that can be problematic for them
\item Hypothesis of Relative Depravation: Prejudice is caused by people being discontent is comparing their current situation to where they expected their situation to be
\end{enumerate}

\subsection{Stigma}
\textbf{Social Stigma:} Disproval or discrimination of a group or individual by society \newline
\textbf{Self Stigma:} Internalizing the stereotypes, prejudice, and discrimination due to a stigma about them leading to individual shame

\section{Social Behavior}
\subsection{Attraction}
\textbf{Proximity Effect:} We are more likely to be attracted to people in close proximity to one another \newline
\textbf{Mere Exposue Effect:} Being exposed to something/someone more often increases our attraction to them \newline
\textbf{Average:} Studies show we are attracted to more average appearing traits than unique ones \newline
\textbf{Similarity Affect:} We are more likely to be attracted to people simmilar to us in terms of both physical and behavioral traits

\subsection{Attachment:}
\textbf{Secure:} Sense of safety, authenticity, and reciprocity \newline
\textbf{Insecure:} Attachment is filled with fear and sense of survival

\subsection{Aggression:}
Physical or verbal behavior intend to harm or destroy based off of 3 infuences:
\begin{enumerate}[nolistsep]
\item Biology: Genes, Impact of brain structure, testosterone
\item Psychological: Frustration aggresion principle where the more frustrated someone is the more likely they are to be aggressive, reinforcement-model as a child where if a child is rewarded for being aggressive or views aggresiveness they tend to be more aggressive
\item Social-Cultural: Deindividuation and social scripts (following a role) can lead to people acting more aggressive
\end{enumerate}


\section{Social Interactions}
\subsection{Terms}
\textbf{Status:} Your role in society \newline
\textbf{Ascribed vs Achieved Status:} A status you were born with versus one you had to earn \newline
\textbf{Role Strain:} A specific role that is providing difficulty or increased stress on an individual \newline
\textbf{Role Conflict:} A conflict between multiple roles an individual holds \newline
\textbf{Primary vs Secondary Group:} People who are close to one another sharing affection vs people who are together to achieve a shared short term goal


\subsection{Altruism}
Acting good when expecting nothing in return however most behavior viewed as altruistic tend to have alterior motives. \newline
\textbf{Kin Selection:} We are more likely to help our kin \newline
\textbf{Reciprocal:} More likely to help someone if we are likely to interact with them in the future \newline
\textbf{Cost Signaling:} As a method to show that you have resources to spare \newline
\textbf{Empathy-Altruism Hypothesis:} More empathetical people are more likely to engage in altruistic behavior

\subsection{Social Support}
\begin{enumerate}[nolistsep]
\item Emotional Support: Affection, trust, love, caring
\item Esteem Support: Expressions of confidence and encouragement
\item Informational Support: Sharing information or wisdom
\item Tangible (Instrumental) Support: Money, items or responsibilities
\item Companionship Support: People who make you feel like you are part of a community
\end{enumerate}




\section{Norms}
\subsection{Types}
\textbf{Folkways:} Common courtesies, being polite \newline
\textbf{Mores:} Morally right and wrong choices without serious concequences\newline
\textbf{Laws:} Rules with punishments \newline
\textbf{Taboos:} Extremely forbidden behaviors viewed as disgusting or wrong

\subsection{Deviance}
Behavior that differs from the norm explained by different theories: 
\begin{enumerate}[nolistsep]
\item Theories of Differential Association: Being surrounded by other deviants
\item Labeling Theory: A behavior is deviant if it was judged that way so some people may take part in deviant behavior without judging it that way themselves. Primary deviance is less severe, secondary deviance are severe with a heavy negative stigma.
\item Strain Theory: Deviant behavior to achieve a socially acceptable goal that the individual is having difficulty achieving. 
\end{enumerate}

\subsection{Collective Behavior}
Short social interactions of groups deviating from social norms:
\begin{enumerate}[nolistsep]
\item Fads: short trends perceived as cool
\item Mass Hysteria: large groups experiencing axiety or fear
\item Riots: collective act of defiance or disapproval
\end{enumerate}




\section{Social Psychology}

\subsection{Group Think}
\textbf{Informative vs Normative Influence:} Informative influence means are influenced due to beleiving as the group does while normative influence means you are influence to not be an outcast \newline
\textbf{Privately vs Publicly Conform:} If you privately conform you change your beliefs to align with the groups, but if you publicly conform you only do it for show while maintaining your actual beliefs in private. \newline
\textbf{Group Think:} Individuals will suppress differences of opinions to maintain group unity. \newline
\textbf{Group Polarization:} Groups tend to make more extreme decisions than any individual would make.

\subsection{Conforimity:}
\textbf{Group Size:} People are more likely to conform if in groups of 3-5 \newline
\textbf{Unaniminity:} People are more likely to conform if group is unanimous. \newline
\textbf{Group Cohesion:} We are more likley to conform if we are more cohesion with the group. \newline
\textbf{Group Status:} If we admire the members of the group we are more likely to conform (ex: trusting a group of doctors) \newline
\textbf{Observed behavior:} More likely to conform if people are observing you vs if you can remain anonymous. \newline
\textbf{Prior commitment:} If people commit to the group they are more likely to conform, but if they denounce it they are less likely to. \newline
\textbf{Feeling of insecurity:} More likely to conform if made to feel insecure.

\subsection{Obedience}
Factors that tend to make us obey:
\begin{enumerate}[nolistsep]
\item Closeness to authority
\item Physical proximity to figure of authority
\item Apparent legitamecy of figure of authority
\item Distance to victim
\item Depersonalization of victim
\item No role model of defiance (everyone else is obeying)
\end{enumerate}

\subsection{Group Effects}
\textbf{Bystander effect:} In the presence of others people are less likely to aid when help is needed due to the diffusion of responsibility theory. \newline
\textbf{Deindividuality:} Individuals in a group are more likely to act impulsively because the presence of the croud conceals the individual's identity. \newline
\textbf{Social Facilition:} Most dominant response is most likely while observed by a group. Improve simple tasks but worsen complex tasks\newline
\textbf{Social Loafing:} People will contribute less to task when in a group. \newline
\textbf{Agents of Socialization:} People or institutions that can impress social norms upon an individual.



\section{Identity and Personality}

\subsection{Evaluation of Self}
\textbf{Self-concept}: Sum of factors in which we describe ourselves (exisential + categorical)\newline
\textbf{Self-esteem}: evaulation of ourselves \newline
\textbf{Self-efficacy}: evalation of ourself to complete a specific task \newline
\textbf{Locus of control}: either internal (our choices matter) or external (outside factors matter) \newline
\textbf{Carl Rogers} beleived that self-concept is made up of self-image, self-esteem, and ideal-self and incongruence is the feeling when your self image does not match your ideal self \newline
\textbf{Social Identity Theory}: People's personal identity is formed partially from their social identity

\subsection{Theories of Development}
\subsubsection{Freud's Stages of Psychosexual Development}

\newcolumntype{P}[1]{>{\centering\arraybackslash}p{#1}}
\noindent \begin{tabular}[H]{|P{1cm}|P{0.75cm}|P{1cm}|P{2cm}|P{1.3cm}|}
\hline
\textbf{Stage} & \textbf{Age} & \textbf{Focus of Libido}& \textbf{Development}& \textbf{Adult Fixation}\\
\hline
Oral & 0-1 & Mouth & Feeding & Smoke, Bite-nails, Over-eat... \\
\hline
Anal & 1-3 & Anus & Toilet Training & Orderliness and Messiness \\
\hline
Phallic & 3-6 & Genital & Oedipus/Electra & Sexual Problems \\
\hline
Latent & 6-12 & N/A & Social Skills & N/A \\
\hline
Genital & 12+ & Genital & Sexual Maturity & Mentally Healthy \\
\hline
\end{tabular}
\newline
\textit{Old Ass People Love Grapefruit}

\subsubsection{Erikson's Stages of Psychosocial Development}
\newcolumntype{P}[1]{>{\centering\arraybackslash}p{#1}}
\noindent \begin{tabular}[H]{|P{0.67cm}|P{0.7cm}|P{1.5cm}|P{1.43cm}|P{1.71cm}|}
\hline
\textbf{Stage} & \textbf{Age} & \textbf{Crisis}& \textbf{Virtue}& \textbf{ (-) Outcome}\\
\hline
1 & 1 & Trust vs Mistrust & Hope & Fear, Suspicion \\
\hline
2 & 2 & Autonomy vs Doubt & Will & Shame \\
\hline
3 & 3-5 & Initiative vs Guilt & Purpose & Inadequacy \\
\hline
4 & 6-12 & Industry vs Inferiority & Competence & Inferiority  \\
\hline
5 & 12-18 & Identity vs Role Confusion & Fidelity & Rebellion \\
\hline
6 & 18-40 & Intimacy vs Isolation & Love & Isolation \\
\hline
7 & 40-65 & Generativity vs Stagnation & Care & Unproductive \\
\hline
8 & 65+ & Integrity vs Despair & Wisdom & Dissatisfaction \\
\hline
\end{tabular}
\newline


\subsection{Theories of Personality}
\textbf{Psychoanalytic:} The cumulation of the id (pleasure seeking impulses), the superego (moral conscious), and ego (the conscious mind trying to gratify the id while satisfying the demands of the id and the moral compass of the super ego).  \newline 
\textbf{Humanistic:} Personality comes from free will and personal growth as people try to reach self actualization. \newline
\textbf{Trait:} Determined by measuring certain traits  \newline
\textbf{Social-Cognitive:} Formed through a mixture of our trait, our interactions with others, and our enviroment \newline
\textbf{Biological:} Determined by genetics and other biological attributes \newline
\textbf{Behavioral:} Learned through stimuli and responses (operant and classical conditioning)\newline
\textbf{Cattell 5 Traits:} Personality is composed of 5 broad factors: conscientiousness, extraversion, neuroticism, openness, and agreeableness

\subsubsection{Psychotherapeutic Approaches}
\textbf{Cognitive-behavioral therapy}: View person as a whole and change their thoughts (cognition) in general \newline
\textbf{Psychoanalytic therapy}: Bring up and resolve unconsious thoughts from the id, ego, superego struggle \newline
\textbf{Humanistic therapy} Help people achieve self actualization, achieve their ideal self\newline

\subsection{Defence Mechanisms}
\textbf{Pathological:} Distort reality ex: denial \newline
\textbf{Immature:} Acting in a socially unacceptable way ex: lashing out \newline
\textbf{Neurotic:} Can lead to anxiety or depression ex: Repression \newline
\textbf{Mature:} Healthy ways to deal with problems ex: humor, altruism... \newline

\subsection{Freud Terms}
\subsubsection{Gratification}
\textbf{Pleasure Princaple:} When we are young or immature we want instant gratification \newline
\textbf{Reality Principle:} Replacing immediate gratification with long term rewards and gratification \newline
\subsubsection{Drives}
\textbf{Eros:} Life drive for health, safety, sex ... \newline
\textbf{Thanatos:} Death drive for fear, anger, hate, selfishness \newline

\section{Learning}
\subsection{Classical Conditioning}
\textbf{Unconditioned Stimulus:} A stimulus that triggers a physiologic/unconditioned response \newline
\textbf{Neutral Stimulus:} An unpaired stimulus
\textbf{Conditioned Stimulus:} A previously neutral stimulus that has now been paired with a unconditioned response (now a conditioned response) \newline 
\textbf{Generalization:} Responding in the same way to different but simmilar stimuli \newline
\textbf{Discrimination:} Responding differently to different stimuli \newline
\textbf{Extinction:} When a conditioned stimulus can no longer stimulate the conditioned response \newline
\textbf{Spontaneous Recovery:} Re-emergence of previously extinct conditioned response

\subsection{Operant Conditioning}

\subsubsection{Terminology}
\textbf{Reinforcement:} Increase the tendency of a goal behavior \newline
\textbf{Punishment:} Decrease the tendency of a behavior occuring again \newline
\textbf{Shaping:} Gradually reinforcing behaviors that comes close to the target behavior \newline
\textbf{Aversive Control:}
\begin{itemize}[nolistsep]
\item Escape Learning: Type of negative reinforcement to distancing oneself from an unpleasent stimulus
\item Avoidance Learning: Escaping an unpleasent stimulus in response to a conditioned stimulus
\end{itemize}


\subsubsection{Reinforcement and Punishment}
\includesvg[scale=0.75]{operant.svg}

\subsubsection{Schedules of Reiforcement}
Partial reiforcement is the when behavior is reinforced only some of the time which is more resilient to extinction then continual reiforcement. 
\begin{itemize}[nolistsep]
\item Fixed ratio: Get a reinforcement after a fixed number of behaviors
\item Variable ratio: Get a reinforcement after a random number of behaviors that averages to a fixed ratio
\item Fixed interval schedule: Reinforce a behavior after a fixed number of times
\item  Variable interval schedule: Reinforcea behavior after a random number of times that averages to a fixed interval schedule
\end{itemize}
Variable reinforcement is more affective than fixed and ratio reinforcement is more affective than interval 

\subsection{Non-Associative Learning}
\textbf{What is it:}Learning that is not associated with a stimulus, reward, or punishment. \newline
\textbf{Sensitization:} Becoming increasingly sensitive to a stimuli heightening the response over time \newline
\textbf{Habituation:} Becoming decreasingly sensitive to a stimuli decreasining the response over time \newline

\subsection{Theories of Learning}
\textbf{Learning-Performance Distintion:} Having learned something is different to perforiming it \newline
\textbf{Bandura's Social Cognitive Theory}
\begin{enumerate}[nolistsep]
\item Attention: Did I pay attention to the lesson
\item Memory: Did I remember the lesson
\item Imitation: Can I imitate the lesson
\item Motivated: Am I motivated to repeat the lesson
\end{enumerate}

\subsection{Behavior}
\textbf{Innate:} Behavior that you know since birth being simple (reflex) or complex (circadian rhythm) \newline
\textbf{Learned:} Behavior that is aquired through habituation, conditioning or insight


\section{Motivation}

\subsection{Theories}
\textbf{Evolutionary:} We do what is needed to survivce \newline
\textbf{Drive-Reduction:} We do what is needed to fulfill our needs \newline
\textbf{Optimum Arousal (incentive):} We do what is needed to be aroused \newline
\textbf{Cognitive Approach:} Our thought process drives behavior \newline
\textbf{Maslow's Hierarchy of Needs:} physiological, safety, love, self-esteem, self-actualization

\subsection{Eating}
\textbf{Ghrelin and Orexin:} Makes you hungry \newline
\textbf{Leptin:} Stops your appetite 

\section{Attitude}
\subsection{What is it?}
An attitude is a learned tendancy composed of three parts:
\begin{enumerate}[nolistsep]
\item Affective: How we feel
\item Behavior: How we behave
\item Cognitive: What we think of something
\end{enumerate}

\subsection{Theories}
How do our attitudes influence our behavior?\newline
\textbf{Theory of Planned Behavior:} We consider our implications and intentions \newline
\textbf{Attitude to Behavior Process Model:} An event triggers an attitude \newline
\textbf{Prototype Willigness Model:} Our behavior is created by our attitudes, our past, our willingness, social norms, our intentions, and our models \newline
\textbf{Elaboration Likelihood Model for Persuasion (ELM):} We get influenced to act certain ways based off of two criteria: 
\begin{enumerate}[nolistsep]
\item Central route: How good of a reason/argument
\item Peripheral route: superficial reasons ex: attractiveness
\end{enumerate}
\textbf{Reciprocal Determinism:} Cognition, enviroment, and behavior are all entertwined and lead to one another. 

\subsection{Behavior influencing Attitude}
\textbf{Foot in the door phenomenon:} tendency to agree to small actions first and will soon comply to do larger actions \newline
\textbf{Role playing:} As we act to fulfill a role it will change our attitude to match it

\subsection{Cognitive Dissonance}
Discomfort experienced when holding conflicting feeling or opnions we handle the situation in 4 ways, people are more likely to change their attitudes than their behaviors.
\begin{enumerate}[nolistsep]
\item Modify: Change your opinion on the topic
\item Trivialize: Change the importance of certain evidence
\item Add: Add additional information to counteract evidence 
\item Deny: Deny the evidence entirely 
\end{enumerate}



\subsection{Persuasion}
\textbf{Message characteristics:} What are the contents of the message \newline
\textbf{Source characteristics:} Is the person delivering the message a good source? \newline
\textbf{Target characteristics:} How are you personally feeling? \newline

\subsection{Control}
\textbf{Locus of Control:}
\begin{enumerate}[nolistsep]
\item Internal: We are responsible for our actions
\item Behavior: External forces are responsible for our actions
\end{enumerate}
\textbf{Learned helplessness:} Becoming helpless as a result of prior experiences out of the individuals control \newline
\textbf{Tyranny of Choice:} Too many choices will lead to decesion paralysis, doubt, and decreased satisfaction in the individuals choice \newline

\subsection{Self Control}
\textbf{Temptation:} A desire that conflicts with long term goals \newline
\textbf{Ego Depletion:} Self control is a limited resource




\section{Cognition}

\subsection{Piaget}
\subsubsection{Stages of Cognitive Development}
\newcolumntype{P}[1]{>{\centering\arraybackslash}p{#1}}
\noindent \begin{tabular}[H]{|P{2cm}|P{2cm}|P{3cm}|}
\hline
\textbf{Stage} & \textbf{Age} & \textbf{Skill}\\
\hline
Sensorimotor & 0-2 & object permanence \\
\hline
Preoperational & 2-7 & pretend play, egocentric \\
\hline
Concrete operational & 7-11 & Conservation, math \\
\hline
Formal & 12+ & Abstract moral reasoning \\
\hline
\end{tabular}
\textit{Some People Can Fly} 

\subsection{Vygostky Theory of Sociocultural Development}
Children develop as a result of social interactions. Starting with 4 elementary functions:
\begin{enumerate}[nolistsep]
\item Attention
\item Sensation
\item Perception
\item Memory
\end{enumerate}
\textbf{More Knowledgable Other:} Individual we learn from to cultivate elementary skills \newline
\textbf{Zone of Proximal Development:} Cognitive area that individual is most sensitive to guidance and when it will be the most effective



\subsubsection{Theories}
\textbf{Schema's:} Experiences, lessons, information... \newline
\textbf{Assimilation:} How we interpret new experiences based off of our schemas \newline
\textbf{Accomadation:} Adapting our schemas to interpret a new one \newline

\subsection{Methods of Problem Solving}
\textbf{Trial and Error:} Guess password randomly 234,537,852 \newline
\textbf{Algorithm Approach:} Try in order 111,112,113 ... \newline
\textbf{Heuristic:} More complex approach
\begin{enumerate}[nolistsep]
\item Means-End analysis
\item Working Backwards
\end{enumerate}

\subsection{Decision Making}
\subsubsection{Heuristics}
\textbf{Availability:} Real examples that come to mind \newline
\textbf{Representativeness:} Matching stereotypes or prototypes \newline
\subsubsection{Bias}
\textbf{Overconfidence:} Things may have felt easy, but you never did it in practice \newline
\textbf{Belief perseverence:} Ignore facts you don't like \newline
\textbf{Confirmation Bias:} Seeking facts that agree with your POV \newline
\textbf{Framing Effect:} Opinion changes based how the problem is framed \newline

\subsection{Intelligence}
\textbf{General Intelligence:} One type of intelligence that encompasses all \newline
\textbf{Primal mental abilities:} Made up of 7 factors \newline
\textbf{Multiple Intelligences:}  7-9 independent intelligences \newline
\textbf{Three Type of Intelligence:} Analytical, creative, practical \newline
\textbf{Emotional Intelligence:} Existence of an emotional intelligence vs a general one \newline
\textbf{Fluid vs Crystalized:} Quick and abstract vs  accumulated knowledge \newline


\section{Language}
\subsection{Neuroanatomy}
\textbf{Hemisphere:} For most people language centers are in the left hemisphere of the brain \newline
\textbf{Broca's Area:} In the left frontal lobe which assists with speech \newline
\textbf{Wernicke's area:} Area of understanding language \newline
\textbf{Arcuate fasciculus:} Links the two area \newline

\subsection{Theories}
\textbf{Universalism:} Thought determines language, we can say what we can think \newline
\textbf{Piaget:} Thought influences language, as we learn our language increases \newline
\textbf{Vygotsky:} Language and thought are independent \newline
\textbf{Linguistic Determinism Weak:} Language influences thought \newline
\textbf{Linguistic Determinism Strong (Whorfian):} Language determines thought \newline


\subsection{Language Development}
\textbf{Nativist/Innatist:} We have a language aquisition device that attunes to a language \newline
\textbf{Learning:} We learn through reinforcement (behaviorist) \newline
\textbf{Interactionist:} Biological + Social factors interact

\section{Emotions}
\subsection{Limbic system}
Limbic system deals with emotional responses and sits on top of the brainstem, HATH \newline
\textbf{Hypothalmus:} Regulates autonomic nervous system \newline
\textbf{Amygdala:} Controls emotions, fear, anxiety, and anger \newline
\textbf{Thalamus:} Sensory relay station \newline
\textbf{Hippocampus:} Forms short term memories into long term memories

\subsection{Parts of Emotion}
\textbf{Three components:} Cognitive, behavioral, and physiological \newline
\textbf{6 Universal Emotions:} Happiness, sadness, fear, disgust, anger, suprise 

\subsection{Theories}
\textbf{James-Lange:} Event $\rightarrow$ Physiologic response  $\rightarrow$ Emotion \newline
\textbf{Cannon-Bard:} Event $\rightarrow$ Emotion + Physiologic response \newline
\textbf{Shakhtar Singer:} Event $\rightarrow$ Physiologic response $\rightarrow$ Interpretation $\rightarrow$ Emotion \newline
\textbf{Lazarus Theory:} Event $\rightarrow$ Interpretation $\rightarrow$ Emotion + Physiological response


\section{Stress}

\subsection{Appraisal of Stress}
\textbf{Primary:} What threat am I experiencing right now which can be irrelevent, benign, or stressful \newline
\textbf{Secondary:} If primary is stressful then move to secondary which encompasses an evaluation of threat (how dangerous is it and what can the individual do) \newline

\subsection{Human's response to stress}
\textbf{4 Major types of stressors:} Significant life change, catastrophe, daily hassles, ambient (not currently significant but long term stress like debt) \newline
\textbf{Two Responces to Stress:} Fight or Flight vs Tend and Befriend

\subsection{General Adaptation Syndrome}
\includesvg[scale=0.4]{GAS.svg}

\section{Senses}
\subsection{Visual Cues}
\textbf{Depth}: Perceived through retinal disparity and convergence (angling of the eyes), relative size, and interposition \newline
\textbf{Form}: Shading and contours \newline
\textbf{Motion}: Motion parallax (relative motion of objects changing with distance) \newline
\textbf{Constancy}: size, shape, and color

\subsection{Threshholds}
\textbf{Weber's Law:} $\dfrac{\Delta I}{I}=K$ \newline
\textbf{Absolute Threshold:} Minimum intensity of a stimulus neededed for it to be detected 50\% of the time \newline
\textbf{Subliminal Stimuli:} Stimuli below absolute threshold of detection \newline
\textbf{Signal detection theory:} The detection of a stimulus depends on both the intensity and traits of the individual. Answer yes a lot is a liberal strategy , while answering no a lot is a conservative strategy. 

\subsection{Somatosensation}

\textbf{Types:} Thermoception (Temperature), Mechanoception (Pressure), Nociception (Pain), Proprioception (Position) \newline
\textbf{Timing:} Non-adapting (constant signal), Slow-adapting (Decreasing signal), Fast-adapting (Signal at start and end only) 


\subsection{Vestibular System}
\textbf{Semicircular Canals:} Composed of three orthogonal canals (anterior, lateral, and posterior) containing endolymph fluid \newline
\textbf{Otolithic organs}: Composed of two organs (utricle and saccule) containing calcium deposits attached to hairs suspended in a fluid

\subsection{Processing}
\textbf{Bottom-up:} Stimulus influences our perception, data drivent \newline
\textbf{Top-Down:}  Uses background knowledge to influence perception, theory driven

\subsubsection{Gestalt's Principles}
\newcolumntype{P}[1]{>{\centering\arraybackslash}p{#1}}
\noindent \begin{tabular}[H]{|P{2cm}|P{5cm}|}
\hline
\textbf{Law} & \textbf{Definition} \\
\hline
Similarity & Similar items are grouped together \\
\hline
Pragnanz & Reality is reduced to simplest form  \\
\hline
Proximity & Objects that are close to one another are grouped together  \\
\hline
Continuity & Lines are seen following the smoothest path \\
\hline
Closure & Objects grouped together to complete a known shape\\
\hline
\end{tabular}
\newline
We tend to view things as a whole rather than individuals. 

\section{Vision}
\subsection{Anatomy}
\includesvg[scale=0.75]{Eye.svg}

\subsection{Physiology of Vision}
\textbf{Rods:} 120 Million, Allows for night vision, Focused around periphery, slow recovery time \newline
\textbf{Cones:} 6-7 Million, Allows to see color, Focused around the fovea, fast recovery time, RGB 60 30 10 \newline

\subsubsection{Feature Detection}
\textbf{Color:} Use cones to determine percentage of RGB \newline
\textbf{Form:} Uses the Parvo pathway which has high spatial resolution (stationary) but low temporal resolution (motion)
\textbf{Motion:} Uses Magno pathway which has high temporal resolution but low spatial resolution, lets us see objects in motion 
\newline
\textbf{Parallel Processing:} The process of using the three prior pathways at the same time

\subsubsection{Phototransduction}
\includesvg[scale=0.6]{Phototransduction.svg} \newline
Light causes a conformational change in retinal resulting it the $\alpha$ subunit of Transducin being released. The $\alpha$ factor activates phosphodiesterase which turns cGMP into GMP. With cGMP no longer available to activate the Na$^{+}$ channels the cell hyperpolarizes. 

\subsubsection{Visual Field Processing}
\includesvg[scale=0.4]{VisualProcessing.svg}


\section{Audition}

\subsection{Anatomy}
\textbf{External Ear:} Pinna, auditory canal, tympanic membrane (eardrum) \newline
\textbf{Middle Ear:} Malleus, incus, and stapes \newline
\textbf{Inner Ear:} Eliptical (oval) window, cochlea, circular (round) window 

\subsubsection{Cochlea in Depth}
\includesvg[scale=0.75]{Cochlea.svg}

\subsection{Basilar Tuning}
\textbf{Place Theory:} Theory that basilar tuning causes that different parts of the basilar membrane respond to difference frequencies \newline
\textbf{Tonotopical Mapping:} Different parts of the primary cortex respond to different frequencies. \newline \newline
\includesvg[scale=0.75]{BasilarTuning.svg}

\section{Somatosensation}
\textbf{Proprioception:} Physicially being able to sense how much each muscle is stretched or relaxed allowing us to know the position of our body in space. \newline
\textbf{Kinesthesia:} Awareness of movement of muscles, for example learning the muscle movements to swing a golf club. \newline
\textbf{Somatosensory homunculus:} Map of the body on the brain in a region called the sensory strip. \newline
\textbf{Adaptation:} The act of down-regulating a signal, for example while your arm is resting on an object don't keep firing action potentials after the arm is at rest since there is no longer a change in pressure. \newline
\textbf{Amplification:} The act of up-regulating a signal, for example when you burn your hand by having one neuron trigger more neurons starting a cascade event. \newline

\section{Olfaction}
\includesvg[scale=0.75]{Olfaction.svg}

\section{Sleep and Conciousness}

\subsection{Brainwaves}
\newcolumntype{P}[1]{>{\centering\arraybackslash}p{#1}}
\noindent \begin{tabular}[H]{|P{3cm}|P{4cm}|}
\hline
\textbf{State of Consciousness} & \textbf{Wave} \\
\hline
Alterness & Beta \\
\hline
Daydreaming & Alpha  \\
\hline
Drowsiness & Theta  \\
\hline
N1 & Theta \\
\hline
N2 & Theta + K Complexes + Sleep Spindles\\
\hline
N3 & Delta\\
\hline
REM & Beta\\
\hline
\end{tabular}
\newline
Sleep Order: N1 $\rightarrow$ N2 $\rightarrow$ N3 $\rightarrow$ N2 $\rightarrow$ REM $\rightarrow$ Repeat 

\subsection{Dreams}
\newcolumntype{P}[1]{>{\centering\arraybackslash}p{#1}}
\noindent \begin{tabular}[H]{|P{2cm}|P{5cm}|}
\hline
\textbf{Theory} & \textbf{Description} \\
\hline
Freud & Dream's have a meaning and are our unconcious urges \\
\hline
Activation Synthesis & Cerebral cortex making sense of random activity from our brain stem  \\
\hline
Evolutionary & Threat simulation, problem solving, or no purpose at all  \\
\hline
\end{tabular}
\newline

\section{Drugs}

\subsection{Types}
\newcolumntype{P}[1]{>{\centering\arraybackslash}p{#1}}
\noindent \begin{tabular}[H]{|P{1.75cm}|P{2.75cm}|P{2.5cm}|}
\hline
\textbf{Drug} & \textbf{Description} & \textbf{Examples} \\
\hline
Depressants & $\downarrow$CNS, $\downarrow$HR, $\downarrow$BP, $\downarrow$Processing Speed & Barbituates, Benzodiazepines, Alcohol \\
\hline
Stimulants & $\uparrow$CNS, $\uparrow$HR, $\uparrow$BP, $\uparrow$Alert & Caffeines, Amphetamines, Nicotine, Cocaine \\
\hline
Hallucinogens & $\uparrow$Sensations, $\uparrow\downarrow$Energy, $\uparrow\downarrow$Mood, hallucinations & LSD, PCP, Psilocybin  \\
\hline
Opiates &  $\downarrow$CNS, $\downarrow$HR, $\downarrow$BP, Analgesic (Pain killer) & Morphine, Heroin, Vicodin  \\
\hline
\end{tabular}
\newline

\subsection{Reward Pathway}
\textbf{Ventral Tegmental Area:} Located in the midbrain and responsible for producing dopamine \newline
\textbf{Hippocampus}: Memory center, will remember the emotion\newline
\textbf{Amygdala}: Processes emotions, will sense that dopamine was positive \newline
\textbf{Prefrontal Cortex}: Processes the experience to help you understand what is happening and what you are enjoying  \newline
\textbf{Nucleus Acumbens}: Helps control motor functions, will help you repeat movement to achieve dopamine again\newline
\textbf{Mesolimbic pathway:} The reward pathways including the regions discussed above HAPN\newline

\section{Attention}
\subsection{Types}
\textbf{Selective:} Focusing on a single topic \newline
\textbf{Divided:} Trying to focus on mutiple topics at once \newline
\textbf{Spotlight Model:} We focus on one task and do not pay attention to other stuff  in the enviroment \newline
\textbf{Resource Model:} We have finite resources availble to commit to paying attention to different things.

\subsection{Theories}
\textbf{Broadbent's Early Selection:} Data goes from sensory register, through a selective filter, then through perceptual processes which assigns meaning. This theory does not explain the cocktail party effect. \newline
\textbf{Deutsch and Deutsch's Late Selection:} Sensory $\rightarrow$ perceptual process $\rightarrow$ selective filter $\rightarrow$ cognitive. This theory claims we perceive everything we sense which is excessive. \newline
\textbf{Treisman's Attentuation Theory:} Sensory $\rightarrow$ attenuator $\rightarrow$ perceptual process $\rightarrow$ cognitive.

\section{Memory}
\subsection{Information Processing Model}
\textbf{1. Sensory memory or register:} Composed of iconic (0.5 seconds) and echoic memory (3-4 seconds) \newline
\textbf{2. Working Memory:} Can hold 7$\pm$2 pieces of information, composed of visual-spatial sketch pad (visual and spatial information) and Phonological loop (verbal information). Central executive coordiates two other components, and when they are combined together they are stored in the episodic buffer. \newline
\textbf{3. Long term memory:} Two main types: explicit and implicit. Explicit memories composed of: semantic (dates) and episodic (birthday party) facts. Implicit memories are composed of procedural (how to ride a bike) and priming (previous experiences that will influence future events) memories.

\subsection{Encoding Strategies}
\textbf{Rote Rehearsal:} Repition \newline
\textbf{Chunking:} Put items of simmilar categories together \newline
\textbf{Mnemonic Devices:} Imagery, pegward (verbal anchors in an order), method of loci (location anchors in order) \newline
\textbf{Acronym:} HAPN (hippocampus, amygdala, pre frontal, nucleus) \newline
\textbf{Self referencing:} Relating new information to you personally \newline
\textbf{Spacing:} Structure studying over time 

\subsection{Retrieval and Memories}
\textbf{Cues:} State (depressed), context (based off of enviroment), priming \newline
\textbf{Free Recall:} Recalling with no cues and just remembering \newline
\textbf{Primacy and Recency effect:} Remembering first and last items on a list respectively \newline
\textbf{Serial Position Effect:} Remembering first and last items well but middle items poorly. \newline
\textbf{Recognition:} Saying an item on a list to see if the person can recognize it \newline
\textbf{Source Memories:} People have difficulty remembering a source of information \newline
\textbf{Flashbulb memory:} Emotional and vivid memories, but still suceptible to reconstruction

\subsection{Cognitive Abilities}
\textbf{Longterm Potentiaion:} Synaptic plasticity makes that some signals become stronger as it repeats, over time process makes it easier to remember some facts \newline
\textbf{Decay:} If a memory is not used over time you will start to forget it \newline
\textbf{Savings:} Makes it easier to relearn something if forgotten \newline
\textbf{Interfence:} Retroactive (new piece of learning interferes with old knowledge) proactive (old piece of knowledge interferes with learning new one)
\textbf{Semantic Network Hierarchy:} We organize things in a logical hierarchy \newline
\textbf{Modified Semantic Network:} We organize things in a experience based hierarchy

\subsection{Aging}
\textbf{Decline:} Recall, episodic memory, processing speed, divided attention \newline
\textbf{Stable:} Implicit memory, recognition \newline
\textbf{Improve:} Semantic memory (until 60), crystallized IQ (using knowledge + experience), emotional reasoning \newline

\section{Biological Basis}

\subsection{Structure of the Brain}

\subsection{Structure of the Nervous System}
\textbf{Central Nervous System:} Brain and Spinal Chord \newline
\textbf{Peripheral Nervous System:} Cranial nerves (nerves that exit the skull), spinal nerverse (nerves that exit the spine) \newline
\textbf{Afferent Neurons:} Bring information to the CNS \newline
\textbf{Efferent Neurons:} Bring information from the CNS \newline
\textbf{Upper vs Lower Motor Neurons:} Upper motor neurons originate in the brain and cross over to the other side of the brain stem around the brain stem then synapse with lower motor neurons in the ventral horn.

\subsection{Autonomic Nervous System}
\subsubsection{Sympathetic Nervous System}
\begin{itemize}[nolistsep]
\item Starts near the middle of the spinal chord
\item Short axon on first neuron which synapses to a second long neuron
\item Will activate fight of flight
\end{itemize}
\subsubsection{Parasympathetic Nervous System}
\begin{itemize}[nolistsep]
\item Starts near the brain stem or bottom of spinal chord
\item Long axon on first neuron which synapses to second short neuron
\item Will activate rest and digest
\end{itemize}

\subsection{Grey and White Matter}
\textbf{Grey Matter:} Consisting mainly of neuron somas \newline
\textbf{White Matter:} Consisting mainly of mylenated axons \newline \newline
\includesvg[scale=0.50]{spinalchord.svg} \newline
In the spinal chord grey matter is located in the middle surround by white matter which is flipped in the brain.

\subsection{Parts of the Brain}
\includesvg[scale=0.75]{Brain.svg} \newline
\textbf{Frontal Lobe:} Motor, prefrontal, Broca's (speech) \newline
\textbf{Parietal Love:} Somatosensory, spatial processing  \newline
\textbf{Occipital Lobe:} Vision \newline
\textbf{Temporal Cortex:} Wernicke's area (language), auditory processing \newline
\textbf{Brain Stem:} Made up of midbrain (relay infor for hearing and vision), pons (sleep-wake, breathing), and medulla (vital processes) \newline
\textbf{Cerebellum:} Coordinates movement \newline
\textbf{Corpus Callosum:} Connects left and right hemisphere \newline
\textbf{Hippocampus:} Forms long term memories \newline
\textbf{Thalamus:} Sesnory relay station \newline
\textbf{Hypothalamus:} Keeps body in homeostasis and manages pituitary gland \newline
\textbf{Basal Ganglia:} Motor control \newline
\textbf{Amygdala:} Process emotions

\subsection{Neurotransmitters}
\textbf{Glutamate:} Released throughout NS, exitatory NT \newline
\textbf{Gamma-Aminobutyric acid:} Released throughout the brain, inhibitory NT \newline
\textbf{Glycine-Aminobutyric acid:} Released throughout the spine, inhibitory NT \newline
\textbf{Acetycholine:} Released throughout lower motor neurons and autonomic nervous sytem \newline
\textbf{Norepinephrine:} Released from the pons and found throughout the brain and autonomic nervous system \newline
\textbf{Histamine:} Released from the hypothalamus \newline
\textbf{Seratonin:} Released throughout the brain \newline
\textbf{Dopamine:} Released from ventral tegmental area and other areas throughout the brain \newline

\subsection{Ways of Studying the Brain}
\textbf{CAT Scan:} Only shows brain structure using xrays \newline
\textbf{MRI:} Only shows brain structure using magnetic fields \newline
\textbf{EEG:} Places electrodes on scalp to read electrical fields giving information on brain function \newline
\textbf{MEG (SQUIDS):} Reads magnetic fields caused by brain, costly expensive machinery \newline
\textbf{fMRI:} Shows MRI image and a heat map showing which parts of brain are active \newline
\textbf{PET Scan:} Combined w/ CAT scan or MRI to create heat map of brain



\section{Mental Disorders}

\subsection{Anxiety Disorders}
\newcolumntype{P}[1]{>{\centering\arraybackslash}p{#1}}
\noindent \begin{tabular}[H]{|P{1.75cm}|P{5.25cm}|}
\hline
\textbf{Disorder} & \textbf{Description} \\
\hline
General Anxiety Disorder & Stress and worry caused by an unclear source relating to more overarching concerns \\
\hline
Panic Disorder & Sudden bursts of panic or fear leading to short bursts of high stress\\
\hline
Phobias & Irrationally afraid of a specific thing or action\\
\hline
Obsessive Compulsive Disorder & Obsession over certain concerns or needs that limit their normal lives \\
\hline
PTSD &  Lingering memories or nightmares of a past event which is affecting their current life\\
\hline
\end{tabular}

\subsection{Medical Symptom Disorders}
\newcolumntype{P}[1]{>{\centering\arraybackslash}p{#1}}
\noindent \begin{tabular}[H]{|P{1.75cm}|P{5.25cm}|}
\hline
\textbf{Disorder} & \textbf{Description} \\
\hline
Somatic Symptom Disorder & Extreme concern relating to one or more physical conditions \\
\hline
Conversion Disorder & Neurological symtoms (paralysis, blindness...) that are not explainable by a medical condition \\
\hline
Illness Anxiety Disorder & Concern with having a serious disease (cancer, HIV...)\\
\hline
Factitious Disorder & Symptoms or illnesses fabricated withouit obvious external gain \\
\hline
\end{tabular}


\end{multicols}
\end{document}