\documentclass[10pt,landscape]{article}
\usepackage{multicol}
\usepackage{calc}
\usepackage{ifthen}
\usepackage[landscape]{geometry}
\usepackage{amsmath,amsthm,amsfonts,amssymb}
\usepackage{color,graphicx,overpic}
\usepackage{tikz}
\usepackage{tikz-3dplot}
\usepackage{pgfplots}
\usetikzlibrary{shapes.geometric}
\usetikzlibrary{decorations.markings}
\usepackage{svg}



\pdfinfo{
  /Title (Bio.pdf)
  /Creator (TeX)
  /Producer (pdfTeX 1.40.0)
  /Author (Kevin)
  /Subject (Biochemistry Cheat Sheet)
  /Keywords (biochemistry, tex, latex)}

\geometry{top=.5in,left=.5in,right=.5in,bottom=.5in}

% Turn off header and footer
\pagestyle{empty}

% Redefine section commands to use less space
\makeatletter
\renewcommand{\section}{\@startsection{section}{1}{0mm}%
                                {-1ex plus -.5ex minus -.2ex}%
                                {0.5ex plus .2ex}%
                                {\normalfont\large\bfseries}}
\renewcommand{\subsection}{\@startsection{subsection}{2}{0mm}%
                                {-1explus -.5ex minus -.2ex}%
                                {0.5ex plus .2ex}%
                                {\normalfont\normalsize\textit}}
\renewcommand{\subsubsection}{\@startsection{subsubsection}{3}{0mm}%
                                {-1ex plus -.5ex minus -.2ex}%
                                {1ex plus .2ex}%
                                {\normalfont\small\underline}}
\makeatother

\renewcommand{\baselinestretch}{1.5}

% Don't print section numbers
\setcounter{secnumdepth}{0}


\setlength{\parindent}{0pt}
\setlength{\parskip}{0pt plus 0.5ex}

%My Environments
\newtheorem{example}[section]{Example}
%

\pgfplotsset{width=6cm,compat=1.9}


\begin{document}
\raggedright
\footnotesize
\begin{multicols}{3}


% multicol parameters
% These lengths are set only within the two main columns
%\setlength{\columnseprule}{0.25pt}
\setlength{\premulticols}{1pt}
\setlength{\postmulticols}{1pt}
\setlength{\multicolsep}{1pt}
\setlength{\columnsep}{2pt}

\begin{center}
     \Large{\textbf{Biology}} \\
\end{center}

\section{DNA}

\subsection{DNA Replication}
$\cdot$ Topoisomerase: Unwinds DNA to be broken in half \newline
$\cdot$ Helicase: Breaks H-Bonds between bases to make two strands \newline
$\cdot$ DNA-Primase: Places RNA primer for DNA-Polymerase \newline
$\cdot$ DNA Polymerase: Builds DNA in 5' to 3' direction \newline
$\cdot$ DNA Ligase: Combines Okazaki fragments and replaces RNA primer w/ DNA \newline

\subsection{Transcription}
\subsubsection{Process}
$\cdot$ Template Strand vs Coding strand: Template strand will used by the RNA polymerase to create a copy of the coding strand \newline
$\cdot$ RNA Polymerase: Attaches to promoter and creates RNA strand in 5' to 3' direction  \newline
$\cdot$ Terminator Region: Signals RNA Polymerase to stop; in prokayotes the RNA forming a hair pin loop is common
\subsubsection{Post Transcription Modification}
$\cdot$ 5' Modified Guanine Cap \newline
$\cdot$ 3' Poly-A tail \newline
$\cdot$ Introns spliced out and exons merged together
\subsubsection{Translation}
$\cdot$ Ribosome sizes: Eukaryotic: 40, 60, 80; Prokaryotic: 30, 50, 70 \newline
$\cdot$ Single Amino acid mass is 110 Daltons \newline \newline
\includesvg[scale=0.75]{Translation.svg}





\end{multicols}
\end{document}