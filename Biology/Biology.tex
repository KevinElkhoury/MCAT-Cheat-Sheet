\documentclass[10pt,landscape]{article}
\usepackage{multicol}
\usepackage{calc}
\usepackage{ifthen}
\usepackage[landscape]{geometry}
\usepackage{amsmath,amsthm,amsfonts,amssymb}
\usepackage{color,graphicx,overpic}
\usepackage{multirow}
\usepackage{array}
\usepackage{tikz}
\usepackage{tikz-3dplot}
\usepackage{pgfplots}
\usetikzlibrary{shapes.geometric}
\usetikzlibrary{decorations.markings}
\usepackage{svg}
\usepackage{enumitem}



\pdfinfo{
  /Title (Bio.pdf)
  /Creator (TeX)
  /Producer (pdfTeX 1.40.0)
  /Author (Kevin)
  /Subject (Biochemistry Cheat Sheet)
  /Keywords (biochemistry, tex, latex)}

\geometry{top=.5in,left=.5in,right=.5in,bottom=.5in}

% Turn off header and footer
\pagestyle{empty}

% Redefine section commands to use less space
\makeatletter
\renewcommand{\section}{\@startsection{section}{1}{0mm}%
                                {-1ex plus -.5ex minus -.2ex}%
                                {0.5ex plus .2ex}%
                                {\normalfont\large\bfseries}}
\renewcommand{\subsection}{\@startsection{subsection}{2}{0mm}%
                                {-1explus -.5ex minus -.2ex}%
                                {0.5ex plus .2ex}%
                                {\normalfont\normalsize\textit}}
\renewcommand{\subsubsection}{\@startsection{subsubsection}{3}{0mm}%
                                {-1ex plus -.5ex minus -.2ex}%
                                {1ex plus .2ex}%
                                {\normalfont\small\underline}}
\makeatother

\renewcommand{\baselinestretch}{1.5}

% Don't print section numbers
\setcounter{secnumdepth}{0}


\setlength{\parindent}{0pt}
\setlength{\parskip}{0pt plus 0.5ex}

%My Environments
\newtheorem{example}[section]{Example}
%

\pgfplotsset{width=6cm,compat=1.9}


\begin{document}
\raggedright
\footnotesize
\begin{multicols}{3}


% multicol parameters
% These lengths are set only within the two main columns
%\setlength{\columnseprule}{0.25pt}
\setlength{\premulticols}{1pt}
\setlength{\postmulticols}{1pt}
\setlength{\multicolsep}{1pt}
\setlength{\columnsep}{2pt}

\begin{center}
     \Large{\textbf{Biology}} \\
\end{center}

\section{DNA}

\subsection{DNA Replication}
$\cdot$ Topoisomerase: Unwinds DNA to be broken in half \newline
$\cdot$ Helicase: Breaks H-Bonds between bases to make two strands \newline
$\cdot$ DNA-Primase: Places RNA primer for DNA-Polymerase \newline
$\cdot$ DNA Polymerase: Builds DNA in 5' to 3' direction \newline
$\cdot$ DNA Ligase: Combines Okazaki fragments and replaces RNA primer w/ DNA \newline

\subsection{Transcription}
\subsubsection{Process}
$\cdot$ Template Strand vs Coding strand: Template strand will used by the RNA polymerase to create a copy of the coding strand \newline
$\cdot$ RNA Polymerase: Attaches to promoter and creates RNA strand in 5' to 3' direction  \newline
$\cdot$ Terminator Region: Signals RNA Polymerase to stop; in prokayotes the RNA forming a hair pin loop is common
\subsubsection{Post Transcription Modification}
$\cdot$ 5' Modified Guanine Cap \newline
$\cdot$ 3' Poly-A tail \newline
$\cdot$ Introns spliced out and exons merged together
\subsubsection{Translation}
$\cdot$ Ribosome sizes: Eukaryotic: 40, 60, 80; Prokaryotic: 30, 50, 70 \newline
$\cdot$ Single Amino acid mass is 110 Daltons \newline \newline
\includesvg[scale=0.75]{Translation.svg}

\section{Skeletal System}
\textbf{Osteoblast:} Forms bone \newline
\textbf{Osteocyte:} Mature Bone Cell \newline
\textbf{Osteoclast:} Breaks down bone \newline
\textbf{Osteon:} Fundamental structural unit of compact bone

\section{Endocrine System}
\subsection{Glands}
\textbf{Hypothalmus:} Control center \newline
\textbf{Pituitary Gland:} Master gland relays center for the hypothalmus getting information from the hypothalmus \newline
\textbf{Thyroid:} Regulates bodies metabolism \newline
\textbf{Parathyroid:} Regulates blood calcium level \newline
\textbf{Adrenal Glands:} Composed of cortex (outer portion) where steroids are made and medulla (inner portion) where catecholamines are made \newline
\textbf{Gonads:} Releases sex hormones \newline
\textbf{Pancreas:} Controls blood sugar 

\subsection{3 Classes of Homones}
\textbf{Autocrine Hormones:} Functions at the cell that makes them \newline
\textbf{Paracrine Hormones:} Function regionally \newline
\textbf{Endocrine Hormones:} Functions at a distance

\subsection{Types of Hormones}
\textbf{Proteins and Polypeptides:} Made in RER and packaged in Golgi into vessicles. Due to them often being polar their receptors are on the cells surface. \newline
\textbf{Steroids:} Derived from lipids most notably cholesterol. All steroids are composed of 4 ring structure (3 6-carbon, and 1 5-carbon). Steroid receptors are typically inside the cell. \newline
\textbf{Tyrosine Derivatives:} Dervied from amino acid tyrosine and can act like steroid or protein based hormones

\subsection{Hormones}
\subsubsection{Hormone Pathways}
\includesvg[scale=0.45]{hormones.svg}

\subsubsection{Anterior Pituitary Gland}
\textbf{FSH:} Regulate growth and regulation \newline
\textbf{LH:} Regulate growth and regulation \newline
\textbf{ACTH:} Regulate adrenal gland \newline
\textbf{TSH:} Regulates thyroid gland \newline
\textbf{Prolactin:} Stimulate lactation and plays a role in other physiological functions \newline
\textbf{Endorphins:} Analgesic effect \newline
\textbf{GH:} Stimulates growth \newline
FLAGPEG

\subsubsection{Posterior Pituitary Gland}
\textbf{Oxytocin:} Controls uterine functions and contraction \newline
\textbf{ADH:} Controls water level

\subsubsection{Gonads}
\textbf{Male:}
\begin{itemize}[]
\item Testosterone production is stimulated by LH
\item Sperm production is stimulated by FSH
\end{itemize}
\textbf{Female:}
\begin{itemize}[]
\item Follicles are stimulated by FSH
\item Estrogen production is triggered by LH
\item Ovulation is triggered by LH
\end{itemize}

\subsubsection{Pineal Gland}
\textbf{Melatonin:} Regulates circadian rhythm

\subsubsection{Adrenal Gland}
\textbf{Adrenal Medulla (Inner):} Catechoalmines (Adrenaline and noradrenaline) controlling fight or flight response  \newline
\textbf{Adrenal Cortex (Outer):}Cortisol and Aldosterone (Steroid hormones) which regulate metabolism and fluid balance

\section{Reproduction}

\subsection{Male}
\subsubsection{Production in 7 steps in the testes:} 
\begin{enumerate}[]
\item Seminiferous tubules
\item Epididymis
\item Vas deferens
\item Ejaculatory Duct
\item Nothing 
\item Urethra
\item Penis
\end{enumerate}
SEVEN UP
\subsubsection{Supporting organs and structures}
\textbf{Seminal vesicles:} Produces fluid mixes with sperm to help protect and nourish \newline
\textbf{Prostate Gland:} Produces fluid that will be mixed with sperm \newline
\textbf{Bulbourethral gland:} Produces fluid that will clean out urethra


\subsection{Female}



\section{Muscles}
\subsection{Categories of Muscles}
\textbf{Skeletal:} Fast straight striated muscles mainly attached to bones and are responsible for skeletal movement \newline
\textbf{Cardiac:} Mid-speed branched striated muscles involved in the heart specialized in synchronized control \newline
\textbf{Smooth:} Slow smooth muscles on walls of hollow organs and blood vessels 

\subsection{Fibers}
\newcolumntype{P}[1]{>{\centering\arraybackslash}p{#1}}
\noindent \begin{tabular}[H]{|P{1.7cm}|P{1.5cm}|P{1.5cm}|P{1.5cm}|}
\hline
\textbf{Fiber} & \textbf{Type 1} & \textbf{Type 2A} & \textbf{Type 2B} \\
\hline
\textbf{Speed} & Slow & Medium & Fast \\
\hline
\textbf{Respiration} & Aerobic & Both & Anaerobic \\
\hline
\textbf{Endurance} & High & Medium & Low \\
\hline
\textbf{Myoglobin} & Many & Medium & Few \\
\hline
\end{tabular}
\newline



\end{multicols}
\end{document}